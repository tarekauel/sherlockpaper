%-----------------------------------------------------------------------------
% Template for seminar 'Program Analysis' at TU Darmstadt.
%
% Adapted from template for sigplanconf LaTeX Class, which is a LaTeX 2e
% class file for SIGPLAN conference proceedings (by Paul C.
% Anagnostopoulos).
%
%-----------------------------------------------------------------------------


\documentclass[authoryear,preprint]{sigplanconf}

% A couple of packages that may be useful
\usepackage{amsmath}
\usepackage{amsfonts}
\usepackage{amssymb}
\usepackage{amsthm}
\usepackage{algorithm2e}
\usepackage{listings}
\usepackage{xcolor}
\usepackage{tikz}
\usepackage{booktabs}
\usepackage{subfigure}
\usepackage[english]{babel}
\usepackage{blindtext}

\usepackage{tikz}
\usepackage{tikz-uml}
\tikzumlset{fill class=white}


\lstdefinelanguage{JavaScript}{
  keywords={typeof, new, true, false, catch, function, return, null, catch, switch, var, if, in, while, do, else, case, break},
  keywordstyle=\color{blue}\bfseries,
  ndkeywords={class, export, boolean, throw, implements, import, this},
  ndkeywordstyle=\color{darkgray}\bfseries,
  identifierstyle=\color{black},
  sensitive=false,
  comment=[l]{//},
  morecomment=[s]{/*}{*/},
  commentstyle=\color{purple}\ttfamily,
  stringstyle=\color{red}\ttfamily,
  morestring=[b]',
  morestring=[b]"
}

\lstset{
   language=JavaScript,
   %backgroundcolor=\color{lightgray},
   extendedchars=true,
   basicstyle=\footnotesize\ttfamily,
   showstringspaces=false,
   showspaces=false,
   numbers=left,
   numberstyle=\footnotesize,
   numbersep=9pt,
   tabsize=2,
   breaklines=true,
   showtabs=false,
   captionpos=b
}


\newcommand{\js}[1]{\lstinline[language=Javascript]$#1$}

\begin{document}

\special{papersize=a4}
\setlength{\pdfpageheight}{\paperheight}
\setlength{\pdfpagewidth}{\paperwidth}


\title{Sherlock: A Profiling Tool to Find Opportunities for Early Property Initialization}

\authorinfo{Tarek Auel \& David G\"usewell}{}{}

\maketitle

\begin{abstract}
\blindtext % replace this with your own text
\end{abstract}


\section{Introduction}
\label{sec:introduction}

%!TEX root = ./main.tex
\section{Goal}

The goal of the developed profiling tool is to find opportunities for early object property and array element initializations as shown in listing \ref{list:example1}. The array \js{a} could be initialized with \js{[1]}. In order to do this the analysis has to keep track of the usage of the elements or properties.

\begin{figure}
\begin{lstlisting}[language=Javascript]
var a = []
a.push(1);
\end{lstlisting}
\caption{Example for early array element initialization.}\label{list:example1}
\end{figure}

If an initialization is conditioned, e.g. in a branch of an \js{if} clause or in the body of a loop, it must not be optimized, because the optimization is known to be potentially invalid for many inputs. Besides, an element or property that has been read cannot be optimized afterwards, because this would change the behavior of the program. But if the scope does allow to do an optimization it has to be done. In listing \ref{list:example4} \js{a} cannot be optimized but \js{b} can.

\begin{figure}
\begin{lstlisting}[language=Javascript]
var a = []
if (cond) {
  a[0] = 1;
  var b = []
  b[0] = 1;
}
\end{lstlisting}
\caption{Example for early array element initialization.}\label{list:example4}
\end{figure}

Many default javascript function do exist that manipulate an array, such as \texttt{reverse}. If a function can be optimized this should be identified by the profiling tool, too. Listing \ref{list:example2} shows an example of unoptimized code (line $1-3$) and the optimized version (line $5-6$).

\begin{figure}
\begin{lstlisting}[language=Javascript]
var a = [1, 2, 3]
a.push(4);
b = a.reverse();

var a = [4, 3, 2, 1]
b = a;
\end{lstlisting}
\caption{Example for an optimaziation.}\label{list:example2}
\end{figure}

Only objects that are initialized by the object notation \js{\{\}} are considered. The objects may define already some properties such as \js{\{a: "name"\}}. Objects that are created using a constructor are not going to be considered, because optimizations may introduce many false positives. Listing \ref{list:example3} shows a constructor with one parameter. Neither \js{this.name = 'bmw'} in line $2$ nor  \js{var c = new car('bmw');} in line $6$ is equal to the original coding. In order to reduce the number of false positives and to limit the scope of the tool, constructors are not taken into account.

\begin{figure}
\begin{lstlisting}[language=Javascript]
function car(name) {
    this.name = name;
    this.knownName = (name !== undefined);
}

var c = new car();
c.name = 'bmw';
\end{lstlisting}
\caption{Example for an optimaziation.}\label{list:example3}
\end{figure}

\begin{itemize}
\item Dynamic program analysis that finds potential opportunities for early initialization of \textbf{object properties} or \textbf{array elements}
\item keep track of every object and array creation
\item track all property lookups
\item track operations that change the state of an object/array, track if conditioned
\end{itemize}
\input{Sherlock.tex}
%! TEX root = main.tex
\section{Evaluation}

\bibliographystyle{abbrvnat}
\bibliography{references}


\bibliographystyle{abbrvnat}



\end{document}
