%-----------------------------------------------------------------------------
% Template for seminar 'Program Analysis' at TU Darmstadt.
%
% Adapted from template for sigplanconf LaTeX Class, which is a LaTeX 2e
% class file for SIGPLAN conference proceedings (by Paul C.
% Anagnostopoulos).
%
%-----------------------------------------------------------------------------


\documentclass[authoryear,preprint]{sigplanconf}

% A couple of packages that may be useful
\usepackage{amsmath}
\usepackage{amsfonts}
\usepackage{amssymb}
\usepackage{amsthm}
\usepackage{algorithm2e}
\usepackage{listings}
\usepackage{xcolor}
\usepackage{tikz}
\usepackage{booktabs}
\usepackage{subfigure}
\usepackage[english]{babel}
\usepackage{blindtext}

\usepackage{tikz}
\usepackage{tikz-uml}
\tikzumlset{fill class=white}
\usepackage{microtype}
\usepackage{graphicx}
\usepackage{adjustbox}
\usepackage{makecell}

\lstdefinelanguage{JavaScript}{
  keywords={typeof, new, true, false, catch, function, return, null, catch, switch, var, if, in, while, do, else, case, break},
  keywordstyle=\color{blue}\bfseries,
  ndkeywords={class, export, boolean, throw, implements, import, this},
  ndkeywordstyle=\color{darkgray}\bfseries,
  identifierstyle=\color{black},
  sensitive=false,
  comment=[l]{//},
  morecomment=[s]{/*}{*/},
  commentstyle=\color{purple}\ttfamily,
  stringstyle=\color{red}\ttfamily,
  morestring=[b]',
  morestring=[b]"
}

\lstset{
   language=JavaScript,
   %backgroundcolor=\color{lightgray},
   extendedchars=true,
   basicstyle=\footnotesize\ttfamily,
   showstringspaces=false,
   showspaces=false,
   numbers=left,
   numberstyle=\footnotesize,
   numbersep=9pt,
   tabsize=2,
   breaklines=true,
   showtabs=false,
   captionpos=b,
   escapeinside=||,
   numberblanklines=false
}


\newcommand{\js}[1]{\lstinline[language=Javascript]$#1$}
\newcommand{\bash}[1]{\lstinline[language=bash]$#1$}

\let\origthelstnumber\thelstnumber
\makeatletter
\newcommand*\Suppressnumber{%
  \lst@AddToHook{OnNewLine}{%
    \let\thelstnumber\relax%
     \advance\c@lstnumber-\@ne\relax%
    }%
}

\newcommand*\Reactivatenumber[1]{%
  \setcounter{lstnumber}{\numexpr#1-1\relax}
  \lst@AddToHook{OnNewLine}{%
   \let\thelstnumber\origthelstnumber%
   \refstepcounter{lstnumber}
  }%
}


\makeatother

\makeatother

\begin{document}

\special{papersize=a4}
\setlength{\pdfpageheight}{\paperheight}
\setlength{\pdfpagewidth}{\paperwidth}


\title{Sherlock: A Profiling Tool to Find Opportunities for Early Property Initialization}

\authorinfo{Tarek Auel \& David G\"usewell}{ tarek.auel@stud.tu-darmstadt.de, davidnikolay.guesewell@stud.tu-darmstadt.de }{}

\maketitle

\begin{abstract}
We present a new dynamic analysis tool for JavaScript, named Sherlock. It uses the  dynamic analysis framework Jalangi. During the analysis Sherlock is able to detect opportunities for early property initialization and give a structured feedback. It is able to handle with many complex program structures like if-conditions or while-loops 
and it support 27 built in JavaScript functions. One main strength of Sherlock is, it keeps the false positives as low as possible. With the proposed optimizations Sherlock can improve the readability of the analyzed code and therefore the maintainability.
\end{abstract}


\section{Introduction}
Within the course Program Testing and Analysis we were given a course project with the title
 "A Profiling Tool to find opportunities for Early Property Initialization". 
 The goal of the project was to design and develop a dynamic program analysis that finds potential  
 opportunities for early initialization of object properties and array elements 
 (at the time of an object/array creation) that are originally added later during the program execution.
 The tool should be able to analyze JavaScript files and use the dynamic analysis framework Jalangi ~\cite{jalangi}.
 Later the developed tool should be tested against Octane ~\cite{octance}, a benchmark for node.js The expected 
 benefits from tool were (i) improved code readability, and (ii) improved performance. In the following section
 we will firstly introduce the goals of our developed tool in a more detailed way. Section~\ref{sec:sherlock} 
 then explain the main techniques behind our Tool Sherlock, how it uses Jalangi and the main capability�s of it.
 The Evaluation of Sherlock is in Section~\ref{sec:evaluation}. Finally Section~\ref{sec:summary} briefly summarize
 our approach.
%!TEX root = ./main.tex
\section{Goal}
\label{sec:goal}
The goal of the developed profiling tool is to find opportunities for early object property and array element 
initializations as shown in listing \ref{list:example1}. The array \js{a} could be initialized with \js{[1]}. 
In order to do this the analysis has to keep track of the usage of the elements or properties.

\begin{lstlisting}[language=Javascript,caption={Example for early array element initialization.},label={list:example1}]
var a = []
a.push(1);
\end{lstlisting}

If an initialization is conditioned, e.g. in a branch of an \js{if} clause or in the body of a loop,
 it must not be optimized, because the optimization is known to be potentially invalid for many inputs.
 Besides, an element or property that has been read cannot be optimized afterwards, because this would 
 change the behavior of the program. But if the scope does allow to do an optimization it has to be done. 
 In listing \ref{list:example4} \js{a} cannot be optimized but \js{b} can.

\begin{lstlisting}[language=Javascript,caption={Example for early array element initialization.},label={list:example4}]
var a = []
if (cond) {
  a[0] = 1;
  var b = []
  b[0] = 1;
}
\end{lstlisting}

Many default javascript function do exist that manipulate an array, such as \texttt{reverse}. 
If a function can be optimized this should be identified by the profiling tool, too. Listing \ref{list:example2}
 shows an example of unoptimized code (line $1-3$) and the optimized version (line $5-6$).

\begin{lstlisting}[language=Javascript,caption={Example for an optimaziation.},label={list:example2}]
var a = [1, 2, 3]
a.push(4);
b = a.reverse();

var a = [4, 3, 2, 1]
b = a;
\end{lstlisting}

Only objects that are initialized by the object notation \js{\{\}} are considered. The objects may define
 already some properties such as \js{\{a: "name"\}}. Objects that are created using a constructor are not going to be considered, 
 because optimizations may introduce many false positives. Listing \ref{list:example3} shows a constructor with one parameter. 
 Neither \js{this.name = 'bmw'} in line $2$ nor  \js{var c = new car('bmw');} in line $6$ is equal to the original coding. 
 In order to reduce the number of false positives and to limit the scope of the tool, constructors are not taken into account.

\begin{lstlisting}[language=Javascript,caption={Example for an optimaziation.},label={list:example3}]
function car(name) {
    this.name = name;
    this.knownName = (name !== undefined);
}

var c = new car();
c.name = 'bmw';
\end{lstlisting}

One of the main goals of Sherlock is to keep the number of false positives as low as possible. 
Given that in some use cases it prefers not optimizing a value instead of running in a potential false positive situation.


\input{Sherlock.tex}
%!TEX root = main.tex
\section{Evaluation}
\label{sec:evaluation}
In order to evaluate the results, it was a given requirement to check, whether Sherlock can find anything that could be optimized in the Octane benchmark. The results of that are presented first, afterwards a general evaluation of the capabilities of Sherlock is given.

\subsection{Performance improvements by Sherlock}
Sherlocks helps finding opportunities for early property or element initialization. Most of the time this improves the readability. Because some statements are discarded, there might be a performance improvement, too.

The following benchmarks has been executed on a MacBook Pro Late 2013 Intel(R) Core(TM) i7-4750HQ CPU @ 2.00GHz. v$0.12.9$ is the used nodejs version. \bash{gdate} is the GNU coreutils implementation of the UNIX \bash{date} function. The command that is used to measure the performance can be found in Listing \ref{list:perf_bash}. We decided to add the loop outside of nodejs in order to not run into just in time compiler optimizations.

\begin{lstlisting}[caption=Performance testing commands,label=list:perf_bash,language=bash]
gdate +%s%3N; for i in {0..1000}; do node optimized.js 2>&1 >/dev/null; done; gdate +%s%3N
\end{lstlisting}

Listing \ref{list:unopt_js_test} shows the unoptimized code. It basically exists of 100 push calls. In order to avoid that JavaScript removes them, the array is printed to \bash{stdout}. The optimized coding in Listing \ref{list:opt_js_test} initializes the array and prints it immediately. Table \ref{tab:perf_results} shows the measured runtimes. The first two lines show the runtime in milliseconds. The second and third line shows the runtime difference. Surprisingly, the unoptimized version is even faster for both number of iterations. The fifth and sixth line show the time that was needed per element. The last two lines show the runtime of the unoptimized version in relation to the optimized one.


\begin{lstlisting}[caption=Unoptimized test coding,label=list:unopt_js_test,language=JavaScript]
var a = [];
a.push(1);
a.push(2);
a.push(3);|\Suppressnumber|
...|\Reactivatenumber{101}|
a.push(100);
console.log(a);
\end{lstlisting}

\begin{lstlisting}[caption=Optimized test coding,label=list:opt_js_test,language=JavaScript]
var a = [
1,
2,
3,|\Suppressnumber|
...|\Reactivatenumber{101}|
100];
console.log(a);
\end{lstlisting}

\begin{table}[h]
\begin{center}
\renewcommand{\thead}[1]{\multicolumn{1}{c}{\bfseries #1}}
\renewcommand{\arraystretch}{1.3}
\begin{tabular}[htbp]{r|r|r}
\thead{\# Iterations} & \thead{Optimized Coding} & \thead{Unoptimized Coding} \\
\hline $1,000$ & $55,086$ ms & $54,831$ ms \\
\hline $10,000$ & $554,431$ ms & $553,333$ ms \\
\hline\hline $1,000$ & $\Delta\ 0$ ms & $\Delta\ -255$ ms \\
\hline $10,000$ & $\Delta\ 0$ ms & $\Delta\ -1,098$ ms \\

\hline\hline $1,000$ & $55.09$ ms/element & $54.83$ ms/element \\
\hline $10,000$ & $55.44$ ms/element & $55.33$ ms/element  \\
\hline\hline $1,000$ & $100\ \%$ runtime & $99.54\ \%$ runtime \\
\hline $10,000$ & $100\ \%$ runtime & $99.80\ \%$ runtime \\ \hline
\end{tabular}
\end{center}
\caption{Measured runtimes in ms including the runtime difference and the runtime per element.}\label{tab:perf_results}
\end{table}

The performance test shows that the optimized code might be even slightly slower. Because of incompatibility issues with Jalangi, Sherlock is forced to use a very old version of nodejs. In newer versions of nodejs the optimized code could be faster.



\subsection{Octance}

Unfortunately Sherlock is not capable to find any optimization in the Octane benchmark. Because there was no obvious reason for that a more detailed analysis was required.

We looked intensively at the \texttt{Richards} benchmark and the \texttt{Splay}. If one analyzes the coding of the benchmarks he can identify quickly why Sherlock could not find any optimization. First of all these benchmark use almost no plain object initialization. Sherlocks does not optimize constructors, because of the reasons mentioned earlier. But plain objects seems to be of almost no interest of the benchmark.
Arrays are used sometimes in the benchmarks. The reason that Sherlock could not find an array optimization is that there are no. If arrays are used they are often initialized with values. If they are initialized as empty arrays, items are often added in loops, conditioned branches, function calls. An optimization can not be applied safely, if it is conditioned. The last thing that one notices while analyzing the benchmark, they do not utilize the default object or array methods. One of the focuses of Sherlock is optimizing built in JavaScript functions. Because the benchmarks do not use them, they could not be found.

In order to check, whether Sherlock is capable to find an optimization in the coding, we instrumented the coding with plain objects and arrays that could be optimized. Listing \ref{list:instru_richards} shows the beginning of the function \js{runRichards} which we instrumented with a couple of statements. Sherlocks correctly determines that \js{a} can be optimized to \js{[1, 6]} and \js{b} to \js{[5, 2, 3]}. This shows that Sherlock can optimize code within benchmarks like Octane.

\begin{lstlisting}[caption=Richards,label=list:instru_richards,language=Javascript]
// ...
function runRichards() {

  var abc = [];
  abc.push(1);

  if (true) {
    // do nothing
    var b = [1, 2, 3];
    while(true) {
      b[1] = 3;
      break;
    }
    b[1] = 3;
    b[0] = 5;
  }

  abc.push(6);

  var scheduler = new Scheduler();
  // ...
\end{lstlisting}

\subsection{Sherlocks capabilities}
Even though Sherlock can cope already with many complex program structures, there are still some patterns known, where Sherlock is either too restrictive or doesn't get them right. 
On of these code patterns is a \js{do-while} loop. The problem can be seen in Listing \ref{lst:dowhile} (line 1 -- 5). The instrumented literal in Line $2$ which tells Sherlock the end of a branching has to be only executed in the second execution of the loop or later. The literal in Line $5$ is used for the last check of \js{cond} which leaves the loop. Even though different patterns exist to mitigate this problem, Sherlock doesn't do this, yet. The first proposal would be to copy all statements and execute them before a \js{while} loop (line 7 -- 12). The second proposal would be to add a counter which masks the literal in line $2$ so that it is not executed in the first iteration (line 14 -- 21). Both solutions do modify the coding more heavily. The counter must have a name that was not used before. This becomes even more tricky if the loops are nested. Is it guaranteed that the semantics may never change, if we copy the statements (1\textsuperscript{st} approach)? Someone could do something that depends on the line number of the coding. If Sherlock copies the coding, there are two different line numbers for the statement. Because of all these difficulties, we decided to ignore this in this stage of development of Sherlock and decide later which Solution we do prefer.

\begin{lstlisting}[label=lst:dowhile,caption=Example for do-while loop and how Sherlock can handle them in future iterations.,language=Javascript]
do {
  'da0b52...ee'; // but only if not 1st loop iteration
  // some statements
} while(cond);
'da0b52...ee';

// some statements
while(cond) {
  // some statements
  'da0b52...ee';
}
'da0b52...ee';

var i = 0;
do {
  if (i != 0) 'da0b52...ee';
  'da0b52...ee'; // literal for the introduced if 
  // some statements
  i++;
} while(cond);
'da0b52...ee';
\end{lstlisting}


All in all Sherlock supports 27 built in JavaScript functions. Due to the fact, that Sherlock tries to report as few false positives as possible it is sometime more restrictive than it has to be. With a deeper tracking of the array elements, it would be possible to allow further optimizations after a \js{Array.prototype.slice} call. But the real world scenarios where this applies are rare. Sherlock can deal with conditioned writes and does know, whether an optimization is safe or not. Even nested structures or loops can be handled. So that, Sherlock does help to improve the readability and therefore the maintainability of coding.











\section{Summary}
The goal of the course project was to develop a new dynamic analysis tool for finding opportunities 
for early property initialization in order to improve the performance and the readability of 
JavaScript code. While there is still a long way to go to find all opportunities, our results 
show that Sherlock is able to cover a broad range of cases which can be optimized. We also give 
some ideas for future iterations of Sherlock to handle do - while loops. Our tool is able to 
automatically generate proposals for early property initialization and keep the number of false 
positive as low as possible in the same time. This is really important for the usability of such a 
dynamic analysis tool.\\ Our results show that the goal to improve the readability of the code can 
be achieved by Sherlock, but there is no improvement in the performance. This may be due the fact
 that we used a very old version of nodejs because Jalangi has some incompability issues with newer versions.  


\label{sec:summary}

\section*{Acknowledgement}
We thank M. Selakovic for the mentoring during the project, K. Sen and L. Gong for Jalangi, A. Hidayat for Esprima and Y.Suzuki for Estraverese and Escodegen.


\bibliographystyle{abbrvnat}
\bibliography{references}





\end{document}
