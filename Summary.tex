\section{Summary}
The goal of the course project was to develop a new dynamic analysis tool for finding opportunities 
for early property initialization in order to improve the performance and the readability of 
JavaScript code. While there is still a long way to go to find all opportunities, our results 
show that Sherlock is able to cover a broad range of cases which can be optimized. We also give 
some ideas for future iterations of Sherlock to handle do - while loops. Our tool is able to 
automatically generate proposals for early property initialization and keep the number of false 
positive as low as possible in the same time. This is really important for the usability of such a 
dynamic analysis tool. So far Sherlock report only the optimized version of an object and cannot
tell the programmer which coding lines could be removed.

Our results show that the goal to improve the readability of the code can 
be achieved by Sherlock, but there is no improvement in the performance. This may be due the fact
 that we used a very old version of nodejs because Jalangi has some incompability issues with newer versions.  


\label{sec:summary}