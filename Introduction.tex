\section{Introduction}
Within the course Program Testing and Analysis we were given a course project with the title
 \glqq{}A Profiling Tool to find opportunities for Early Property Initialization\grqq{}. 
 The goal of the project was to design and develop a dynamic program analysis that finds potential  
 opportunities for early initialization of object properties and array elements 
 (at the time of an object/array creation) that are originally added later during the program execution.
 The tool should be able to analyze JavaScript files and use the dynamic analysis framework Jalangi ~\cite{jalangi}.
 Later the developed tool should be tested against Octane ~\cite{octance}, a benchmark for node.js The expected 
 benefits from tool were (i) improved code readability, and (ii) improved performance.

 In the following section
 we will firstly introduce the goals of our developed tool in a more detailed way. Section~\ref{sec:sherlock} 
 then explain the main techniques behind our tool Sherlock, how it uses Jalangi and the main capabilities of it.
 The Evaluation of Sherlock is in Section~\ref{sec:evaluation}. Finally Section~\ref{sec:summary} briefly summarize
 our approach.